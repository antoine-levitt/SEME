

\section{Introduction}

Micrométéorites,
%écailles de peinture détachée d'un engin spatial, 
petites pièces relâchées lors de la séparation de deux étages d'une fusée, 
débris produits par la collision d'un vieux satellite avec d'autres débris... 
l'espace en orbite basse est plein de petits objets, qui représentent un danger permanent pour le matériel et pour les hommes qui s'y trouvent.
Leur danger vient de leur vitesse, 7 ou 8~km/s, et de leur nombre : la \textsc{nasa} estime à plus de 21~000 le nombre de débris de plus de 10~cm, et à plus de de 500~000 les débris ayant entre 1 et 10~cm de diamètre\footnote{Source : \url{http://orbitaldebris.jsc.nasa.gov/faqs.html}. Notons qu'un tiers de tous les débris ont été créés par la destruction volontaire d'un vieux satellite par un missile chinois, en 2007, et par la collision accidentelle d'un satellite américain et d'un satellite russe en 2009.}. Il est donc éminemment nécessaire de pouvoir les détecter.

Ces débris peuvent être repérés depuis le sol. Mais comment faire pour en détecter le plus possible, tout en minimisant naturellement le coût pour le faire ? Tel est le problème que Max Cerf (\textsc{eads}--Astrium) nous a posé, lors de la 4\ieme{} 
Semaine d'étude mathématiques et entreprises (\textsc{seme}), qui s'est tenue à
dans les locaux de l'Institut Henri Poincaré (\textsc{ihp}) à Paris du lundi 15
au vendredi 19 octobre 2012.  Plusieurs aspects contribuent à rendre ce problème difficile :
\begin{enumerate}
\item Les débris évoluant sur des orbites différentes, la latitude du télescope influe fortement sur le nombre d'orbites traversées par le champ de vision ;
\item Le nombre de débris que l'on peut observer sur une orbite donnée dépend aussi du temps pendant lequel cette orbite reste visible lorsqu'elle passe dans le champ de vision ;
\item Pour pouvoir observer un débris, encore faut-il qu'il soit éclairé par la lumière du soleil, et que le détecteur ne soit pas lui-même ébloui, c'est-à-dire qu'il faut qu'il fasse nuit pour lui ;
\item Enfin, la période de révolution d'un débris modifie la probabilité qu'il puisse être observé, suivant qu'il est plus ou moins synchronisé avec le détecteur.
\end{enumerate}

Les deux premiers aspects ont été étudiés dans la section~2, le troisième dans la section~3, le dernier dans la section~4. Dans la section~5, des simulations numériques donnent des résultats directs, indépendamment de la modélisation des aspects précités.
\paragraph*{Remerciements}

Le groupe souhaite remercier les organisateurs de la \textsc{seme}, ainsi que Astrium et plus précisément Max
Cerf pour avoir proposé ce sujet, que nous avons trouvé riche et intéressant.