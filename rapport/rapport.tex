\documentclass[a4paper,11pt]{article}
\usepackage{cmap}
\usepackage{textcomp}
\usepackage[french]{babel}
\frenchbsetup{og=«,fg=»}
\usepackage[utf8]{inputenc}
\usepackage[T1]{fontenc}
\usepackage{lmodern}
\usepackage{graphicx}
\usepackage{color}
\usepackage{hyperref}
\usepackage{eurosym}
\usepackage{upgreek}
\usepackage{mathrsfs}
\usepackage{amssymb}
\usepackage{amsmath}
\usepackage{graphicx}
\usepackage{amsthm}
\usepackage{pgf,tikz}
\usetikzlibrary{arrows}
\usepackage{enumerate}
\usepackage{amsmath, amssymb, amsthm}
\usepackage{url}
\usepackage{multicol}
\newcommand{\eps}{\varepsilon}
\renewcommand{\leq}{\leqslant}
\renewcommand{\geq}{\geqslant}

\numberwithin{section}{part}
\renewcommand{\thesection}{\arabic{section}}

\setlength{\textwidth}{15cm} \setlength{\hoffset}{-1.50cm}
\setlength{\textheight}{630pt} \setlength{\voffset}{-1.4cm}

%\selectlanguage{french}

\begin{document}

%% Debut couverture
%%%%% NICOLAS 1/2 %%%%%
\let\oldtextsc\textsc
\let\textsc\uppercase
\newcommand{\affil}[1]{\up{#1}}
%%%%%%%%%%%%%%%%%
 \setcounter{page}{0}
 \thispagestyle{empty}
\begin{center}
\sffamily
 {\LARGE \sc Semaine d'\'Etude Maths--Entreprises 4} \\
%\vspace*{0.5cm}
\vfill
 {\large  15--19 octobre 2012, Institut Henri Poincaré (Paris)}
\end{center}
\vfill
\begin{center}
 \LARGE \textsf{Observation de débris spatiaux}
\end{center}
\vspace*{0.5cm}
\begin{center}
 \large \sffamily
\begin{tabular}{cc}
         Nicolas \textsc{Bonnotte}\affil{a} & Maxime \textsc{Chupin}\affil{b} \\
         Tony \textsc{Février}\affil{a} & Antoine \textsc{Levitt}\affil{c} \\ 
         Benjamin \textsc{Marteau}\affil{d} & Vladimir \textsc{Salnikov}\affil{e}
   \end{tabular}
\end{center}
\vfill
% chez moi, l'usage de l'italique rend de nouveau les points actifs dans \url{}, et les fait donc suivre d'un espace insécable (!)
\centerline{\footnotesize \affil{a} Université Paris-Sud, \url{nicolas.bonnotte@math.u-psud.fr}}
\centerline{\footnotesize \affil{b} Université Pierre et Marie Curie, \url{chupin@ann.jussieu.fr}}
\centerline{\footnotesize \affil{c} Université Paris Dauphine, \url{levitt@ceremade.dauphine.fr}}
\centerline{\footnotesize \affil{d}  Université de Versailles Saint-Quentin-en-Yvelines, \url{benjamin.marteau@laposte.net}}
\centerline{\footnotesize \affil{e} \textsc{insa} de Rouen, \url{vladimir.salnikovinsa-rouen.fr}}

% \begin{multicols}{2}
% \participant{Nicolas Bonnotte}{doctorant à l'Université Paris-Sud}{nicolas.bonnotte@math.u-psud.fr}
% \participant{Maxime Chupin}{}{}
% \participant{Tony Février}{}{}
% \participant{Antoine Levitt}{doctorant à l'Université Paris Dauphine}{levitt@ceremade.dauphine.fr}
% \participant{Benjamin Marteau}{}{}
% \participant{Vladimir Salnikov}{}{}
% % rajoutez vos affiliations :
% % \participant{Prénom nom}{Affiliation}{Adresse électronique}
% \end{multicols}

\vfill

\hspace{1cm}Sujet propos\'e par :
\begin{center}
\begin{tabular}{cc}
\includegraphics[scale=.4]{LOGO_EADS_ASTRIUM.jpg} 
\end{tabular}
\end{center}
\begin{center}
  \large \sffamily Correspondant : Max \textsc{Lecerf} (EADS)
\end{center}
\vfill
\vspace{3cm}
\includegraphics[width=0.27\linewidth]{logo_GDR_ME} \hfill
\includegraphics[width=0.12\linewidth]{logo_CNRS} \hfill
\includegraphics[width=0.27\linewidth]{logo_AMIES} \hfill
\includegraphics[width=0.30\linewidth]{logo_LMV}
%% Fin couverture
%%%%% NICOLAS 2/2 %%%%%
\let\textsc\oldtextsc
%%%%%%%%%%%%%%%%%

\newpage
 \setcounter{page}{0}
\thispagestyle{empty}
\begin{center}

  \abstract{ La prolifération des débris spatiaux est un danger pour
    les satellites artificiels. On étudie ici le problème de
    l'observation de ces débris : où faut-il placer des téléscopes sur
    terre et combien en faut-il pour observer la plupart des débris
    dangereux ? Cette étude est compliquée par les conditions
    d'observation des débris : on ne peut les voir que lorsqu'ils sont
    éclairés par le soleil, et que le téléscope ne l'est pas. On
    calcule la probabilité d'observation d'un débris en un jour (aux
    alentours de $1\%$), qu'on vérifie par des simulations
    numériques. On étudie enfin plus précisément l'indépendance des
    observations d'un jour sur l'autre, ce qui conduit à la conclusion
    qu'il est possible que des débris passent entre les mailles du
    filet avec une probabilité bien plus grande que ce à quoi on
    s'attendrait si les observations étaient indépendantes. Cela
    tendrait à montrer qu'il vaut mieux utiliser deux téléscopes
    pendant un an qu'un téléscope pendant deux ans.}
  
  \vfill

  Mots clés : mécanique céleste, observation de débris spatiaux,
  conditions d'éclairement.

\vfill
Num\'ero de publication : SEME00X-201X-0X-X
\end{center}
\newpage

%
%\section{Contexte}
%
%Ce travail a été réalisé dans le cadre de la 4\ieme{} 
%Semaine d'étude mathématiques et entreprises (\textsc{seme}), qui s'est tenue à
%dans les locaux de l'Institut Henri Poincaré (\textsc{ihp}) à Paris du lundi 15
%au vendredi 19 octobre 2012. Le groupe souhaite remercier les
%organisateurs de la \textsc{seme}, ainsi que Astrium et plus précisément Max
%Cerf pour avoir proposé ce sujet, que nous avons trouvé intéressant et
%riche.

% \begin{multicols}{2}
% \participant{Nicolas Bonnotte}{doctorant à l'Université Paris-Sud}{nicolas.bonnotte@math.u-psud.fr}
% \participant{Maxime Chupin}{}{}
% \participant{Tony Février}{}{}
% \participant{Antoine Levitt}{doctorant à l'Université Paris Dauphine}{levitt@ceremade.dauphine.fr}
% \participant{Benjamin Marteau}{}{}
% \participant{Vladimir Salnikov}{}{}
% % rajoutez vos affiliations :
% % \participant{Prénom nom}{Affiliation}{Adresse électronique}
% \end{multicols}



%\section{Introduction}



\section{Introduction}

Micrométéorites,
%écailles de peinture détachée d'un engin spatial, 
petites pièces relâchées lors de la séparation de deux étages d'une fusée, 
débris produits par la collision d'un vieux satellite avec d'autres débris... 
l'espace en orbite basse est plein de petits objets, qui représentent un danger permanent pour le matériel et pour les hommes qui s'y trouvent.
Leur danger vient de leur vitesse, 7 ou 8~km/s, et de leur nombre : la \textsc{nasa} estime à plus de 21~000 le nombre de débris de plus de 10~cm, et à plus de de 500~000 les débris ayant entre 1 et 10~cm de diamètre\footnote{Source : \url{http://orbitaldebris.jsc.nasa.gov/faqs.html}. Notons qu'un tiers de tous les débris ont été créés par la destruction volontaire d'un vieux satellite par un missile chinois, en 2007, et par la collision accidentelle d'un satellite américain et d'un satellite russe en 2009.}. Il est donc éminemment nécessaire de pouvoir les détecter.

Ces débris peuvent être repérés depuis le sol. Mais comment faire pour en détecter le plus possible, tout en minimisant naturellement le coût pour le faire ? Tel est le problème que Max Cerf (\textsc{eads}--Astrium) nous a posé, lors de la 4\ieme{} 
Semaine d'étude mathématiques et entreprises (\textsc{seme}), qui s'est tenue à
dans les locaux de l'Institut Henri Poincaré (\textsc{ihp}) à Paris du lundi 15
au vendredi 19 octobre 2012.  Plusieurs aspects contribuent à rendre ce problème difficile :
\begin{enumerate}
\item Les débris évoluant sur des orbites différentes, la latitude du télescope influe fortement sur le nombre d'orbites traversées par le champ de vision ;
\item Le nombre de débris que l'on peut observer sur une orbite donnée dépend aussi du temps pendant lequel cette orbite reste visible lorsqu'elle passe dans le champ de vision ;
\item Pour pouvoir observer un débris, encore faut-il qu'il soit éclairé par la lumière du soleil, et que le détecteur ne soit pas lui-même ébloui, c'est-à-dire qu'il faut qu'il fasse nuit pour lui ;
\item Enfin, la période de révolution d'un débris modifie la probabilité qu'il puisse être observé, suivant qu'il est plus ou moins synchronisé avec le détecteur.
\end{enumerate}

Les deux premiers aspects ont été étudiés dans la section~2, le troisième dans la section~3, le dernier dans la section~4. Dans la section~5, des simulations numériques donnent des résultats directs, indépendamment de la modélisation des aspects précités.
\paragraph*{Remerciements}

Le groupe souhaite remercier les organisateurs de la \textsc{seme}, ainsi que Astrium et plus précisément Max
Cerf pour avoir proposé ce sujet, que nous avons trouvé riche et intéressant.


%\section{Observation du débris}

\renewcommand{\phi}{\varphi}

\newcommand{\TD}{T_\text{d}}
\newcommand{\omegaD}{\omega_\text{d}}
\newcommand{\iD}{i_\text{d}}
\newcommand{\altitudeD}{r}
\newcommand{\rayonD}{\rayonT+\altitudeD}

\newcommand{\TT}{T_\text{t}}
\newcommand{\omegaT}{\omega_\text{t}}
\newcommand{\iT}{i_\text{t}}
\newcommand{\rayonT}{R}

\newcommand{\referentiel}{\mathcal{R}}

\section{Observabilité d'une orbite}

\subsection{Référentiel et notations}

Supposons qu'au temps $t=0$, on puisse voir au centre du télescope une certaine orbite, et donc potentiellement les débris qui s'y trouveraient. Pendant combien de temps pourra-t-on encore la voir ? Et, quand elle aura disparue, pendant combien de temps l'aura-t-on vue au total ?

On se place dans un référentiel $\referentiel = (O,\vec{e}_x,\vec{e}_y,\vec{e}_z)$ dont le centre est le centre de la Terre, tel que $(O,\vec{e}_x,\vec{e}_y)$ soit le plan de l'équateur. Notons $\iT \in [-\frac{\pi}{2}, \frac{\pi}{2}]$ l'inclinaison du télescope vis-à-vis de l'équateur, et $\phi \in [0,\pi]$ sa latitude : $\iT = \frac{\pi}{2} - \phi$, et $\phi=0$ correspond au pôle Nord, $\phi = \pi$ au pôle Sud. En notant  $\rayonT$ le rayon de la Terre et $\omegaT = \frac{2\pi}{\TT}$ la pulsation du télescope, les coordonnées de ce dernier dans notre référentiel $\referentiel$  sont :
\[ 
\overrightarrow{OT} = \rayonT \vec{u} \qquad \text{avec} \qquad \vec{u}(t) = \left(\begin{array}{c}
\cos(\omegaT t) \cos(\iT) \\
\sin(\omegaT t) \cos(\iT) \\
\sin(\iT)
\end{array}
\right).
\]
Supposons que l'orbite soit un cercle de centre $O$, de rayon $\rayonD$, dans un plan incliné d'un angle $\iD \in [0, \frac{\pi}{2}]$ vis-à-vis de l'équateur et de direction normale
\[ 
\vec{n} =  \left(\begin{array}{c}
 \sin(\iD)\cos(\theta) \\
\sin(\iD)\sin(\theta) \\
 \cos(\iD) 
\end{array}
\right).
\]
%L'orbite se trouve donc dans le plan $(O, \vec{u}, \vec{v})$, avec
%\[ 
%\vec{u} =  \left(\begin{array}{c}
% -\sin(\theta) \\
%\cos(\theta) \\
%0 
%\end{array}
%\right)
%\qquad \text{et} \qquad
%\vec{v} = \vec{n} \wedge \vec{u} =  \left(\begin{array}{c}
% -\cos(\iD) \cos(\theta) \\
% -\cos(\iD)\sin(\theta) \\
%  \sin(\iD)
%\end{array}
%\right)
%\]
À $t=0$, le télescope coupe le plan de l'orbite, et cela entraîne $\vec{n}\cdot \vec{u} = 0$. L'angle $\theta$ doit donc vérifier la relation
\[ \cos(\theta) = - \frac{\sin(\iT)\cos(\iD)}{\cos(\iT)\sin(\iD)}.\]

On note $\alpha$ le demi-angle de d'ouverture du télescope, ou plutôt le demi-angle qui serait celui sous lequel un observateur placé au centre de la Terre verrait la même chose à l'altitude $\altitudeD$ que le télescope. Si $\alpha_0$ est l'ouverture réelle du télescope, on a (voir \autoref{nicolas:alpha}) :
\[ \alpha = \frac{\rayonT}{\rayonD} \alpha_0 \approx 0,1\text{\degres}.\]

\begin{figure}
\begin{center}
\scriptsize
\def\figurewidth{0.6\linewidth}

\definecolor{xdxdff}{rgb}{0.49,0.49,1}
\definecolor{qqwuqq}{rgb}{0,0.39,0}
\definecolor{qqqqcc}{rgb}{0,0,0.8}
\definecolor{ffqqqq}{rgb}{1,0,0}
\definecolor{qqqqff}{rgb}{0,0,1}
\begin{tikzpicture}[line cap=round,line join=round,>=triangle 45,x=1.0cm,y=1.0cm]
\clip(4.11,-0.63) rectangle (10.8,5.62);
\draw [shift={(5,3)},color=qqwuqq,fill=qqwuqq,fill opacity=0.1] (0,0) -- (49.79:1.16) arc (49.79:90:1.16) -- cycle;
\draw [shift={(5,0)},color=qqwuqq,fill=qqwuqq,fill opacity=0.1] (0,0) -- (72.58:1.02) arc (72.58:90:1.02) -- cycle;
\draw [color=ffqqqq] (5,0) circle (5cm);
\draw [color=qqqqcc] (5,0) circle (3cm);
\draw (5,5)-- (5,0);
\draw (5,0)-- (6.5,4.77);
\draw (5,3)-- (6.5,4.77);
\draw (4.62,4.12) node[anchor=north west] {$$r$$};
\draw (4.56,1.84) node[anchor=north west] {$$R$$};
\begin{scriptsize}
\fill [color=qqqqff] (5,0) circle (1.5pt);
\fill [color=qqqqff] (5,5) circle (1.5pt);
\fill [color=qqqqff] (5,3) circle (1.5pt);
\draw[color=qqqqff] (5.49,3.17) node {Téléscope};
\fill [color=qqqqff] (6.5,4.77) circle (1.5pt);
\draw[color=qqwuqq] (5.77,3.58) node {$\tilde \alpha$};
\draw[color=qqwuqq] (5.4,0.63) node {$\alpha$};
\fill [color=xdxdff] (5.75,4.94) circle (1.5pt);
\draw[color=xdxdff] (6.07,5.11) node {Débris};
\end{scriptsize}
\end{tikzpicture}

\caption{Rapport entre les angles $\alpha$ et $\alpha_0$} \label{nicolas:alpha}
\end{center}
\end{figure}


\subsection{Calcul approché de la durée journalière d'observation}

Supposons que l'orbite apparaisse dans la visée du télescope avec un angle $\psi$ vis-à-vis de l'horizontale. Elle balaye la fenêtre d'observation avec une vitesse $\omegaT \sin(\psi)$ (voir \autoref{nicolas:fenetre}). Il lui faut donc un temps $\tau_0$ pour traverser entièrement celle-ci, avec $\tau_0$ donné par la formule
\[ \tau_0 = \frac{2\alpha}{\omegaT \sin(\psi)}.\]
Cet angle $\psi$ est l'angle entre le vecteur vitesse du télescope $T$ et celui d'un débris $D$ situé sur l'orbite. 
%On connait déjà $\overrightarrow{OT}$ ; quant à $\overrightarrow{OD}$, on sait qu'il se déplace dans un plan $(O, \vec{f}_1, \vec{f}_2)$ perpendiculaire à $\vec{n}$ sur un cercle de rayon $\rayonD$ de centre $O$ et de vitesse angulaire $\omegaD = \frac{2\pi}{\TD}$, et que $O$, $T$ et $D$ sont alignés pour $t=0$. Ceci nous suffit pour obtenir :
%\[ \overrightarrow{OD} = (\rayonD)\vec{v} \qquad \text{avec} \qquad \vec{v} = \cos(\omegaD t) \vec{f}_1 + \sin(\omegaD t) \vec{f}_2\]
%si du moins
%\[ 
%\vec{f}_1 =  \left(
%	\begin{array}{c}
%		\cos(\iT) \\
%		0\\
%		\sin(\iT)
%	\end{array}
%\right) \quad \text{et} \quad \vec{f}_2 =  \frac{1}{\cos(\iT)} \left(
%	\begin{array}{c}
%		\sin(\iT) \sqrt{\sin(\iD)^2 - \sin(\iT)^2} \\
%		\cos(\iD) \\
%		-\cos(\iT) \sqrt{\sin(\iD)^2 - \sin(\iT)^2}
%	\end{array}
%\right).
%\]
%Par conséquent, l'angle $\psi$ est donné par la formule
On peut montrer que
\[ 
\cos(\psi) = \frac{\cos(\iD)}{\cos(\iT)}.
\]
Puisqu'il y a deux observations par jour, la durée journalière d'observation de l'orbite est
\[ \tau = 2\tau_0 = \frac{4\alpha\cos(\iT)}{\sqrt{\sin(\iD)^2 - \sin(\iT)^2}}.\]
Malheureusement, cette formule n'a probablement pas de sens si $\iD$ et $\iT$ sont trop rapprochés, ou si $\iT$ est trop proche de $\frac{\pi}{2}$. C'est pourquoi nous allons maintenant essayer une autre méthode, exacte celle-ci.

\begin{figure}
\begin{center}
\scriptsize
\def\figurewidth{0.6\linewidth}
%\newlength{\unit}
%\newlength{\width}
\setlength{\width}{\figurewidth}
\setlength{\unit}{0.3\width}



\definecolor{qqzzqq}{rgb}{0,0.6,0}
\definecolor{ffqqqq}{rgb}{1,0,0}
\definecolor{qqqqcc}{rgb}{0,0,0.8}
\definecolor{uququq}{rgb}{0.25,0.25,0.25}

\begin{tikzpicture}[line cap=round,line join=round,>=triangle 45,x=\unit,y=\unit]
\draw[color=black, dashed] (-1,0) -- (1,0);
%\foreach \x in {-1,-0.5,0.5,1}
%\draw[shift={(\x,0)},color=black] (0pt,2pt) -- (0pt,-2pt) node[below] {\footnotesize $\x$};
%\draw[color=black] (0,-1.36) -- (0,1.28);
%\foreach \y in {-1,-0.5,0.5,1}
%\draw[shift={(0,\y)},color=black] (2pt,0pt) -- (-2pt,0pt) node[left] {\footnotesize $\y$};
%\draw[color=black] (0,0.125) node[right] {\footnotesize $O$};
\clip(-1.44,-1.36) rectangle (1.31,1.28);
\draw [shift={(-1.25,-1)},color=qqqqcc,fill=qqqqcc,fill opacity=0.1] (0,0) -- (0:0.18) arc (0:75.07:0.18) -- cycle;
\draw [->] (-1.25,-1) -- (-0.85,-1);
\draw [->] (-1.25,-1) -- (-1.05,-0.25);
\draw [color=red,domain=-.486:0.0133] plot(\x,{(--0.19--0.75*\x)/0.2});
\draw [color=red,densely dotted,domain=-0.0368:-.5295] plot(\x,{(--0.19--0.75*(\x+0.05))/0.2});
\draw [color=red,dotted,domain=-0.0876:-.5720] plot(\x,{(--0.19--0.75*(\x+0.1))/0.2});
\draw [color=red,loosely dotted,domain=-.13926:-.6138] plot(\x,{(--0.19--0.75*(\x+0.15))/0.2});
%\draw [color=qqzzqq,domain=-1.44:1.31] plot(\x,{(-0-0.2*\x)/0.75});
\draw [->, color=red] (-0.43,-0.67) -- (-0.06,-0.77);
%\draw [->] (-0.43,-0.67) -- (-0.2,0.18);
\begin{scriptsize}
\fill [color=black] (0,0) circle (1.5pt);
%\draw[color=uququq] (0.05,0.08) node {$O$};
\draw[color=black] (-0.87,-0.92) node {$\omegaT$};
\draw[color=black] (-1,-0.61) node {$\omegaD$};
\draw[color=qqqqcc] (-1.15,-0.92) node {$\psi$};
\draw[color=black] (0.13,-0.66) node {$\omegaT \sin(\psi)$};
%\draw[color=black] (0.15,-0.3) node {$\omegaD + \omegaT \cos(\psi)$};
\draw [line width=1pt] (0,0) circle (1);
\end{scriptsize}
\end{tikzpicture}
\caption{Fenêtre d'observation} \label{nicolas:fenetre}
\end{center}
\end{figure}



\subsection{Calcul exact de la durée journalière d'observation}

On rappelle que $\alpha$ désigne le demi-angle d'observation en équivalent pour un observateur situé au centre de la Terre. L'orbite est donc visible si et seulement si
\[ -\sin(\alpha) \leq \vec{n} \cdot \vec{u} \leq \sin(\alpha).\]
Notons $t_\pm$ les deux instants les plus proches de $t=0$ tels que
\[  \vec{n} \cdot \vec{u}(t_+) = \sin(\alpha) \qquad \text{et} \qquad  \vec{n} \cdot \vec{u}(t_-) = -\sin(\alpha).\] 
Puisque $\alpha$ est petit, on a donc :
\begin{multline*}
\cos(\omegaT t_\pm) \cos(\iT) \sin(\iD) \cos(\theta) + \sin(\omegaT t_\pm) \cos(\iT)\sin(\iD)\sin(\theta)
 = \pm\alpha -  \sin(\iT)\cos(\iD).
\end{multline*}
Ceci entraîne
\[ \cos(\theta - \omegaT t_\pm) = \cos(\theta) \pm  \frac{\alpha}{\cos(\iT)\sin(\iD)}.\]
Puisque $t_\pm$ est petit, 
\[ \sin(\theta) \omegaT t_\pm - \frac{1}{2} \cos(\theta) \left(\omegaT t_\pm\right)^2 = \pm  \frac{\alpha}{\cos(\iT)\sin(\iD)},\]
et on obtient donc
\[ \omegaT t_\pm = A - \sqrt{A^2 \mp \frac{\alpha\cos(\iT)\sin(\iD)}{\sin^2(\iT)\cos^2(\iD)}} \qquad \text{avec} \qquad A=\frac{\cos(\iT)\sin(\iD)\sqrt{\sin^2(\iD)-\sin^2(\iT)}}{\sin^2(\iT)\cos^2(\iD)}\]
Le taux d'observation sur une journée est  $ \tau = 2(t_+ - t_-)$, en convenant que $t_\pm = 0$ lorsque la formule donnant $t_\pm$ n'a pas de sens. On a :
\[ \tau = \frac{2}{\omegaT} \left(  \sqrt{A^2 + \frac{\alpha\cos(\iT)\sin(\iD)}{\sin^2(\iT)\cos^2(\iD)}} - \sqrt{A^2 - \frac{\alpha\cos(\iT)\sin(\iD)}{\sin^2(\iT)\cos^2(\iD)}} \right).\]
En particulier,
\begin{align*}
  \tau =  &\frac{2}{\omegaT}\sqrt{\frac{\alpha}{\cos(\iD)\sin(\iD)}} & \text{si $\iD = \iT$}, \\
  \tau =  &\frac{2}{\omegaT}\frac{\alpha}{\sqrt{\sin(\iD)^2 - \sin(\iT)^2}} & \text{si $\iD > \iT \gg \alpha$}.
\end{align*}%

\begin{figure}
\scriptsize
\begin{center}
\def\svgwidth{0.5\linewidth}
 \input{figures/observabilite3.pdf_tex}
 \vspace*{-2em}
\caption{Durée journalière d'observation (en minutes)} \label{nicolas:observabilite}
\end{center}
\end{figure}%

La \autoref{nicolas:observabilite} montre le nombre de minutes par jour où l'orbite est visible. Cette durée ne dépend que de $\iT$ et de $\iD$. Lorsque $\iD=\iT=0$ ou $\iD=\iT=\frac{\pi}{2}$, l'orbite est en permanence visible : le premier cas correspond au télescope situé sur l'équateur avec une orbite équatoriale, le second au télescope au pôle Nord avec une orbite polaire. Lorsque $\iD < \iT$, l'orbite n'est pas assez inclinée et n'est donc jamais visible par le télescope, situé lui à une latitude trop élevée. Lorsque $\iD = \frac{\pi}{2}$ et $\iT = \frac{\pi}{4}$, c'est-à-dire pour une orbite polaire et un télescope à la latitude de Bordeaux, l'orbite est visible 96 secondes par jour. Lorsque $\iD = \frac{\pi}{2}$ et $\iT = 0$, c'est-à-dire pour une orbite polaire et un télescope à l'équateur, l'orbite n'est plus visible que 48 secondes par jour. 


\subsection{Partie de l'orbite observée chaque jour}

Lorsque l'on voit l'orbite durant un temps $\mathrm{d}t$, celle-ci tournant avec une vitesse $\omegaD$, on observe en fait une proportion $\mathrm{d}p = \frac{\omegaD}{2\pi} \mathrm{d} t = \frac{\mathrm{d}t}{\TD}$. La proportion de l'orbite observée par jour est donc
\[ p = \frac{\omegaD \tau}{2\pi} = \frac{\tau}{\TD}.\]
Si $\TD = 1 \text{h}$, pour une orbite polaire et un télescope à l'équateur on observe $p \approx 1 \%$ de l'orbite par jour. En supposant donc que l'on n'observe jamais deux fois les mêmes débris, et que ceux-ci sont répartis uniformément  sur l'orbite, on peut donc espérer observer $1 \%$ des débris par jours.







\section{Conditions d'\'eclairement}

Dans cette partie on se place dans un rep\`ere g\'eocentrique qui tourne \`a la vitesse de rotation de la Terre autour du soleil. Ainsi on peut voir le mouvement du soleil relativement \`a la Terre comme une simple variation de l'incidence $i_S$ de ses rayons.  La \autoref{ombre} repr\'esente ce rep\`ere : la Terre est en rouge, la sph\`ere transparente est celle contenant le d\'ebris et en bleu, on observe le c\^one d'ombre g\'en\'er\'e par les rayons incidents. 

 \begin{figure}[ht]
    \centering
    \includegraphics[width=.6\textwidth]{figures/ombre.pdf}
    \caption{C\^one d'ombre}\label{ombre}
 \end{figure}
 
Ainsi les saisons sont d\'etermin\'ees par une variation de l'angle d'incidence entre le plan \'equatorial et les rayons solaires.
On mod\'elise dans notre rep\`ere cette variation par 
\[ i_{S}(t)= i_{max}\cos(\omega_{s}t), \quad \text{avec} \quad i_{max}=23.75^\circ \quad \text{et}  \quad \omega_s=\frac{2\pi}{T_s}.\]
 
L'objectif de cette partie est de d\'eterminer la proportion de ciel observable en prenant en compte les contraintes suivantes : 
\begin{itemize}
\item le d\'ebris doit \^etre \'eclair\'e par le soleil ;
\item le t\'elescope doit \^etre dans la nuit.
\end{itemize}
Ces deux contraintes s'int\`egrent au mod\`ele par l'interm\'ediaire de deux angles d\'ependant du temps $t$ : $\beta(t)$ repr\'esentant le c\^one d'ombre auquel ne doit pas appartenir le d\'ebris, $\delta(t)$ permettant de d\'elimiter la nuit du jour . Afin de mieux voir o\`u interviennent ces deux angles, on s'int\'eresse \`a deux coupes horizontales repr\'esent\'ees en rouge dans la \autoref{coupehor}:
\begin{itemize}
\item celle incluant la trajectoire du t\'elescope (coupe de latitude $\varphi$) ;
\item celle contenant le cercle des points o\`u le d\'ebris sera observable par le t\'elescope (les intersections entre les droites du z\'enith du t\'elescope et la sph\`ere de motion du d\'ebris).
\end{itemize}
\begin{figure}
\begin{center}
\scriptsize
\def\figurewidth{0.6\linewidth}
% \documentclass[10pt]{article}
% \usepackage[utf8]{inputenc}
% \usepackage{pgf,tikz}
% \usetikzlibrary{arrows}
% \pagestyle{empty}
% \begin{document}
\definecolor{qqwuqq}{rgb}{0,0.39,0}
\definecolor{ffqqqq}{rgb}{1,0,0}
\definecolor{xdxdff}{rgb}{0.49,0.49,1}
\definecolor{qqqqff}{rgb}{0,0,1}
\definecolor{uququq}{rgb}{0.25,0.25,0.25}
\begin{tikzpicture}[line cap=round,line join=round,>=triangle 45,x=1.0cm,y=1.0cm,scale=0.10]
\clip(-26.49,-38.19) rectangle (51.52,18.74);
\draw [shift={(0,0)},color=qqwuqq,fill=qqwuqq,fill opacity=0.1] (0,0) -- (0:3.39) arc (0:27.02:3.39) -- cycle;
\draw(0,0) circle (10cm);
\draw(0,0) circle (15.95cm);
\draw [domain=-26.49:51.52] plot(\x,{(--100-8.91*\x)/4.54});
\draw [domain=-26.49:51.52] plot(\x,{(-0--4.54*\x)/8.91});
\draw [color=ffqqqq,domain=-26.49:51.52] plot(\x,{(-7.24-0*\x)/-1});
\draw [color=ffqqqq,domain=-26.49:51.52] plot(\x,{(-4.54-0*\x)/-1});
\draw [domain=-26.49:51.52] plot(\x,{(-0-0*\x)/-10});
\draw (0,-38.19) -- (0,18.74);
\begin{scriptsize}
\fill [color=uququq] (0,0) circle (1.5pt);
\draw[color=uququq] (0.56,0.91) node {$A$};
\fill [color=qqqqff] (10,0) circle (1.5pt);
\draw[color=qqqqff] (10.45,0.91) node {$B$};
\fill [color=qqqqff] (14.21,7.24) circle (1.5pt);
\draw[color=qqqqff] (14.72,8.1) node {$D$};
\fill [color=xdxdff] (8.91,4.54) circle (1.5pt);
\draw[color=xdxdff] (9.43,5.45) node {$T$};
\draw[color=qqwuqq] (3.13,0.64) node {$\varphi$};
\end{scriptsize}
\end{tikzpicture}
% \end{document}

\caption{Rapport entre les angles $\alpha$ et $\alpha_0$} \label{coupehor}
\end{center}
\end{figure}
Ces deux coupes horizontales sont celles des figures \ref{coupehorbeta} et \ref{coupehordelta}. Ainsi le t\'elescope doit non seulement \^etre hors du jour (il lui reste $\pi-2\delta(t)$ de p\'erim\`etre disponible) mais aussi il ne doit pas pointer dans le c\^one d'ombre (enlever $2\beta(t))$ de p\'erim\`etre.
\begin{figure}
\begin{center}
\scriptsize
\def\figurewidth{0.6\linewidth}

\definecolor{qqwuqq}{rgb}{0,0.39,0}
\definecolor{uququq}{rgb}{0.25,0.25,0.25}
\definecolor{xdxdff}{rgb}{0.49,0.49,1}
\definecolor{qqqqff}{rgb}{0,0,1}
\begin{tikzpicture}[line cap=round,line join=round,>=triangle 45,x=1.0cm,y=1.0cm,scale=0.7]
\draw[->,color=black] (-5.76,0) -- (6.47,0);
\foreach \x in {-5,-4,-3,-2,-1,1,2,3,4,5,6}
\draw[shift={(\x,0)},color=black] (0pt,2pt) -- (0pt,-2pt) node[below] {\footnotesize $\x$};
\draw[->,color=black] (0,-4.44) -- (0,3.78);
\foreach \y in {-4,-3,-2,-1,1,2,3}
\draw[shift={(0,\y)},color=black] (2pt,0pt) -- (-2pt,0pt) node[left] {\footnotesize $\y$};
\draw[color=black] (0pt,-10pt) node[right] {\footnotesize $0$};
\clip(-5.76,-4.44) rectangle (6.47,3.78);
\draw [shift={(0,0)},color=qqwuqq,fill=qqwuqq,fill opacity=0.1] (0,0) -- (0:0.7) arc (0:22:0.7) -- cycle;
\draw(0,0) circle (3cm);
\draw(0,0) circle (2cm);
\draw [domain=-5.76:6.47] plot(\x,{(-0-0*\x)/2});
\draw (2.78,-4.44) -- (2.78,3.78);
\draw (0,0)-- (2.78,1.12);
\draw (0,0)-- (2.78,-1.12);
\begin{scriptsize}
\fill [color=qqqqff] (0,0) circle (1.5pt);
\draw[color=qqqqff] (-0.12,0.24) node {$O$};
\fill [color=xdxdff] (2,0) circle (1.5pt);
\draw[color=xdxdff] (2.1,0.16) node {$T$};
\fill [color=uququq] (0,2) circle (1.5pt);
\draw[color=uququq] (0.2,2.16) node {$N_1$};
\fill [color=uququq] (3,0) circle (1.5pt);
\draw[color=uququq] (3.1,0.16) node {$D$};
\fill [color=uququq] (0,-2) circle (1.5pt);
\draw[color=uququq] (0.09,-1.84) node {$S$};
\fill [color=xdxdff] (2.78,1.12) circle (1.5pt);
\draw[color=xdxdff] (2.88,1.28) node {$A$};
\fill [color=uququq] (2.78,-1.12) circle (1.5pt);
\draw[color=uququq] (2.87,-0.96) node {$B$};
\draw[color=qqwuqq] (0.56,0.11) node {$\beta$};
\end{scriptsize}
\end{tikzpicture}
\caption{Rapport entre les angles $\alpha$ et $\alpha_0$} \label{coupehorbeta}
\end{center}
\end{figure}
\begin{figure}
\begin{center}
\scriptsize
\def\figurewidth{0.6\linewidth}

\definecolor{qqwuqq}{rgb}{0,0.39,0}
\definecolor{uququq}{rgb}{0.25,0.25,0.25}
\definecolor{xdxdff}{rgb}{0.49,0.49,1}
\definecolor{qqqqff}{rgb}{0,0,1}
\begin{tikzpicture}[line cap=round,line join=round,>=triangle 45,x=1.cm,y=1.cm,scale=0.6]
\draw[->,color=black] (-5.76,0) -- (6.47,0);
\foreach \x in {-5,-4,-3,-2,-1,1,2,3,4,5,6}
\draw[shift={(\x,0)},color=black] (0pt,2pt) -- (0pt,-2pt) node[below] {\footnotesize $\x$};
\draw[->,color=black] (0,-4.44) -- (0,3.78);
\foreach \y in {-4,-3,-2,-1,1,2,3}
\draw[shift={(0,\y)},color=black] (2pt,0pt) -- (-2pt,0pt) node[left] {\footnotesize $\y$};
\draw[color=black] (0pt,-10pt) node[right] {\footnotesize $0$};
\clip(-5.76,-4.44) rectangle (6.47,3.78);
\draw [shift={(0,0)},color=qqwuqq,fill=qqwuqq,fill opacity=0.1] (0,0) -- (63.02:0.64) arc (63.02:90:0.64) -- cycle;
\draw(0,0) circle (3cm);
\draw (0,0)-- (1.36,2.67);
\draw (1.36,-4.44) -- (1.36,3.78);
\draw (0,0)-- (1.36,-2.67);
\begin{scriptsize}
\fill [color=qqqqff] (0,0) circle (1.5pt);
\draw[color=qqqqff] (-0.12,0.24) node {$O$};
\fill [color=xdxdff] (1.36,2.67) circle (1.5pt);
\draw[color=xdxdff] (1.47,2.84) node {$A$};
\fill [color=uququq] (1.36,-2.67) circle (1.5pt);
\draw[color=uququq] (1.46,-2.51) node {$B$};
\fill [color=uququq] (0,3) circle (1.5pt);
\draw[color=uququq] (0.11,3.17) node {$C$};
\draw[color=qqwuqq] (0.26,0.39) node {$\delta$};
\end{scriptsize}
\end{tikzpicture}
\caption{Rapport entre les angles $\alpha$ et $\alpha_0$} \label{coupehordelta}
\end{center}
\end{figure}
La proportion de zone observable par le t\'elescope est donc donn\'ee par 
\[{p_{V}(t,\varphi)=\frac{\pi-2\beta(t)-2\delta(t)}{2\pi}}.\]
Des arguments de g\'eom\'etrie \'elementaires  nous permettent d'obtenir
\[ \beta(t)=\arccos\left(\frac{\sqrt{2rR+r^2}\cos(i_s(t))}{(R+r)\sin(\varphi)}\right) \qquad \text{et} \qquad \delta(t)=\arcsin\left(\frac{|\tan(i_s(t))|}{\tan(\varphi)}\right).\]
 o\`u $R$ et $R+r$ d\'esignent respectivement le rayon de la Terre et de l'orbite du d\'ebris.
Munis de ces r\'esultats,  nous avons ainsi pu d\'eterminer pour chaque latitude $\varphi$ la moyenne de proportion de zone observable sur un an, le but \'etant de d\'eterminer la latitude id\'eale de placement du t\'elescope. 

 \begin{figure}[ht]
    \centering
    \includegraphics[width=.7\textwidth]{opt.png}
    \caption{Proportion de zone observable pendant un an}\label{opt}
 \end{figure}
 
 On observe deux tendances dans le graphique de la proportion (\ref{opt}) : une premi\`ere partie ou celle-ci d\'ecroit (de la latitude $0$ \`a environ 0.4 radians) et une seconde ou elle cro\^it. Cette transition de phase correspond \`a la latitude du cercle polaire arctique. Il semble donc qu'il vaille mieux placer le t\'elescope au plus pr\`es possible des p\^oles.
 Ce choix se doit d'\^etre nuanc\'e par la \autoref{ecart} : on y trace les extremas de la proportion de zone observable en fonction de la latitude \`a laquelle on se trouve.
 On constate que dans les zones les plus \'eclair\'ees (au-dessus du cercle polaire arctique selon le graphique pr\'ec\'edent), il y a des p\'eriodes au cours desquelles il fait exclusivement nuit.
 On n'observe donc rien durant ce laps de temps.
 
  \begin{figure}[ht]
    \centering
    \includegraphics[width=.7\textwidth]{ecart.png}
    \caption{Extremas de la proportion de zone observable}\label{ecart}
 \end{figure}


\section{Indépendance des observations}
Les calculs que nous avons faits donnent la probabilité $p$ qu'étant
donné un débris de paramètres $a_{D}$ et $i_{D}$ positionné
aléatoirement sur son orbite, il soit vu en un jour. La question qui
se pose maintenant est celle de l'indépendance des
observations. Peut-on considérer que l'observation au jour $J+1$ est
indépendante de l'observation au jour $J$ ? Si on regarde le ciel
pendant $N$ jours, le nombre de fois où le débris est vu suit-il une
loi binomiale $B(n,p)$, comme ce serait le cas si les observations
étaient indépendantes ? Si non, peut-on obtenir des estimations sur la
durée d'observation nécessaire pour voir le débris ?

La réponse mathématique à ces questions est délicate. Dans cette
partie, on introduit un modèle simplifié qui reprend néanmoins les
caractéristiques principales du problème réel, et on étudie le passage
à un modèle probabiliste. On considère un téléscope et un débris, et
on ignore les problèmes de visibilité dûs au soleil, ainsi que le
mouvement de précession.

Si la latitude du téléscope le permet, la zone d'observation du
téléscope croise l'orbite du débris deux fois par jour, tous les jours
aux mêmes heures $t_{1}$ et $t_{2}$. Pendant tout le temps où l'orbite
est visible, le débris parcourt un angle $\beta$ sur son orbite. Dans
le régime qui nous intéresse, $\beta \approx 10 \alpha$ : on
considérera donc l'approximation $\beta \gg \alpha$ où le téléscope
observe un point de l'orbite, et verra le débris si il traverse ce
point. Entre deux croisements de l'orbite et du téléscope, le débris
parcourt un angle $\theta$ sur son orbite. $\theta$ alterne entre deux
valeurs $\theta_{t_{1} \to t_{2}}$ et $\theta_{t_{2} \to t_{1}}$. Ces
deux valeurs ne varient pas de jour en jour (même si on prend en
compte la précession). On simplifie le modèle en considérant que
$\theta_{t_{1} \to t_{2}} = \theta_{t_{2} \to t_{1}} = \theta$, ou de
façon équivalente qu'on n'observe qu'une fois sur les deux
opportunités par jour. Cette simplification n'affecte pas les
caractéristiques principales du modèle.

% \begin{center}
%   \definecolor{qqwuqq}{rgb}{0,0.39,0}
\definecolor{ffqqqq}{rgb}{1,0,0}
\definecolor{qqqqff}{rgb}{0,0,1}
\begin{tikzpicture}[line cap=round,line join=round,>=triangle 45,x=1.0cm,y=1.0cm]
\clip(-2.46,-1.58) rectangle (5.82,5.82);
\draw [shift={(1,2)},color=qqwuqq,fill=qqwuqq,fill opacity=0.1] (0,0) -- (0:2) arc (0:74.05:2) -- cycle;
\draw [shift={(1,2)},color=qqwuqq,fill=qqwuqq,fill opacity=0.1] (0,0) -- (143.2:2) arc (143.2:161.57:2) -- cycle;
\draw [color=ffqqqq] (1,2) circle (3.16cm);
\draw (1,2)-- (-2,3);
\draw (1,2)-- (-1.53,3.89);
\draw (1,2)-- (4.16,2);
\draw (1,2)-- (1.87,5.04);
\begin{scriptsize}
\fill [color=qqqqff] (1,2) circle (1.5pt);
\fill [color=black] (4.16,2) circle (1.5pt);
\draw[color=black] (5.0,2.26) node {$\text{Débris à t}$};
\fill [color=black] (1.87,5.04) circle (1.5pt);
\draw[color=black] (3.54,5.3) node {$\text{Débris à }t+Tobs$};
\draw[color=qqwuqq] (2.22,2.6) node {$\theta$};
\draw[color=qqwuqq] (0.64,2.48) node {$\alpha$};
\end{scriptsize}
\end{tikzpicture}

%   \input{ergodic}
  
%   (refaire meilleure figure)
% \end{center}

On pose
\begin{align}
  \label{modelejouet}
  \theta_{n} = (\theta_{0} + n \theta) \mod 1,
\end{align}
où $\theta_{0}$ est une phase initiale aléatoirement distribuée, et
$\theta$ est un paramètre. On considère qu'on a observation du débris
au jour $n$ si $0 \in [\theta_{n}, \theta_{n}+\beta]$, où $\beta$ est
la fraction de l'orbite parcourue par le débris pendant une
journée. On appelle $N(n)$ le nombre cumulé d'observations du débris
au jour $n$. On sait que si $\theta$ est irrationnel, alors
\begin{align}
  \label{largenumbers}
  \lim_{n\to\infty} \frac{N(n)} n = \beta.
\end{align}
(théorème d'équirépartition\footnote{\url{http://en.wikipedia.org/wiki/Equidistribution_theorem}})

Ce théorème est un analogue de la loi des grand nombres : en temps
grand, la fréquence d'observation du débris est égale à la proportion
du cercle observée par le téléscope. Le problème est que ce théorème
ne nous apporte pas d'informations concrètes sur la vitesse de
convergence de cette limite, qui nous permettrait d'obtenir des
conclusions pratiques. 

Si au lieu du modèle déterministe \eqref{modelejouet} on utilise un
modèle probabiliste où $\theta_{n}$ est une variable aléatoire
distribuée uniformément entre 0 et 1, alors $N(n)$ suit la loi
binomiale $B(n,\beta)$. Cela implique par exemple
\begin{align}
  P(N(n) \geq 1) = 1 - (1-\beta)^{n}.
\end{align}

La question est de savoir si des estimations semblables peuvent être
obtenues pour le modèle déterministe. Ce problème est un sujet d'étude
en théorie ergodique. Un théorème de Kesten\cite{kesten} montre que si
$\theta$, $\theta_{0}$ sont distribués uniformément sur $[0,1]$ et
$\beta$ est irrationnel, alors le nombre $N(n)$ d'observations en $n$
itérations suit pour $n$ grand la loi
\begin{align}
  N(n) \approx n \beta + {\rho} \log n\; \mathcal C,
\end{align}
où $\rho$ est une constante universelle\footnote{le théorème est
  également valable pour $\beta$ rationnel, mais alors le facteur
  $\rho$ dépend non-trivialement de $\beta$}, et $\mathcal C$ est la
distribution de Cauchy, de densité
\begin{align}
  f_{\mathcal C}(x) = \frac 1 \pi \frac 1 {1+x^{2}}
\end{align}

La distribution de Cauchy a des traînes extrêmement épaisses (sa
moyenne n'est même pas définie). En particulier, le théorème implique
que
\begin{align}
  P(N(n) = 0) \approx_{n\to\infty} \frac {\rho}{\beta \pi}\frac{\log n}{n}
\end{align}

Cette décroissance en $\frac {\log n} n$ est extrêmement lente (elle
est à comparer avec la décroissance exponentielle en $(1-p)^{n}$ si
les observations étaient indépendantes d'un jour à l'autre).

% Les simulations numériques que nous avons effectuées montrent que
% l'approximation aléatoire est une bonne approximation, mais ne permet
% pas de rendre compte de certains phénomènes (certains débris mettent
% plus de temps à se laisser observer que dans le modèle aléatoire). La
% question de la compréhension fine de ce phénomène est une question
% ouverte intéressante du point de vue à la fois théorique (théorie
% ergodique) et pratique (pour se prémunir de la possibilité de ne pas
% observer certains débris aussi rapidement que ce que prédit l'approche
% probabiliste).

%\section{Simulation numérique}


\section{Simulation num\'erique}

\subsection{Syst\`eme \'etudi\'e}
Pendant les simulations num\'eriques au cas g\'en\'erique on est capable 
d'\'etudier le syst\`eme de $N$ satellites (d\'ebris) observ\'es par $M$
t\'elescopes. Pour cela on utilise les solutions connues des \'equations 
de mouvement: 
% \begin{center}
% \includegraphics[width = 0.75\linewidth]{map2.pdf} 
% \end{center}
\begin{eqnarray}
   \theta_{O_i}(t) &=& \theta_{O_i}(t_0) + (t-t_0)\dot \theta_{O_i}  \nonumber \\
   \theta_{D_i}(t) &=& \theta_{D_i}(t_0) + (t-t_0)\dot \theta_{D_i} , \, i = 1, \dots N \label{vladimir:sat} \\
   \theta_{T_j}(t) &=& \theta_{T_j}(t_0) + (t-t_0)\dot \theta_{T_j} , \, j = 1, \dots M \label{vladimir:tel} \\
   \theta_S(t) &=& \theta_S(t_0) +  (t-t_0)\dot\theta_S  \label{vladimir:sun} 
\end{eqnarray}
Ici les deux premi\`eres \'equations sur $\theta_{O_i}$ et
$\theta_{D_i}$ caract\'erisent la position du $i$'\`eme satellite par
rapport \`a un rep\`ere fixe; la troisi\`eme \'equation sur
$\theta_{T_j}$ est expliqu\'ee par la rotation de la Terre et
caract\'erise donc la position des t\'elescopes; la derni\`ere
\'equation d\'efinit l'évolution de la direction du Soleil.

La trajectoire typique du satellite (dans un rep\`ere fixe) va « colorer » une partie de
la sph\`ere de rayon correspondant \`a l'altitude de satellite, ou plus pr\'ecis\'ement la sph\`ere
sans les « chapeaux » autour des p\^oles, donn\'es par l'angle d'inclinaison d'orbite
(voir \autoref{vladimir:repfx})
 \begin{figure}[htp] \centering
     \includegraphics*[width=3in, height=3in]{repfx.png}
      \caption{
            \label{vladimir:repfx}
Trajectoire de satellite dans un syst\`eme des coordonn\'ees fixe. }
 \end{figure}

 %% Note de Antoine : la figure est incompréhensible sans les axes, je
 %% la commente ...


 
% Sa projection sur la Terre tournante est donn\'ee sur la \autoref{vladimir:mks}. 
% On a l'habitude de voir les trajectoires comme celle-ci dans les
% centres de contr\^ole des missions spatiales.
%  Comme on le voit clairement, la trajectoire g\'en\'erique n'est pas p\'eriodique
%  (\autoref{vladimir:mks}, zone entourée).


% \begin{figure}[htp] \centering
%       \includegraphics*[height=3in]{mks.png}
%       \caption{
%             \label{vladimir:mks}
% Proj\'ection de traj\'ectoire de satellite}
%  \end{figure}
 

\subsection{La loi de probabilit\'e d'observation}
La premi\`ere s\'erie des tests num\'eriques a pour but de v\'erifier
l'hypoth\`ese de la loi de probabilit\'e d'observer un satellite 
donn\'e avec un t\'elescopes donn\'e. 
Comme \`a chaque instant on connait la position du satellite et du t\'elescope
(équations (\ref{vladimir:sat}, \ref{vladimir:tel})), on peut comparer ses coordonn\'ees sph\'eriques 
et conclure si le satellite est dans le champ de vue de t\'elescope.

% Sur la \autoref{vladimir:mks} cette situation correspond \`a l'intersection de la projection de la
% trajectoire avec la zone bleue, repr\'esentant le t\'elescope.
% (L'image est bien sûr sch\'ematique: pour la vraie taille du champ de vue de t\'elescope
% utilis\'ee dans les simulation, voir la section sur l'observabilit\'e d'une orbite.)

Le test num\'erique est donc une simulation de mouvement de $N$ satellites
tous sur l'orbite avec le m\^eme angle d'inclinaison $i_d$ de la valeur proche (mais pas \'egale) 
\`a la valeur correspondante aux satellites h\'eliosynchrones, mais avec une l\'eg\`ere diff\'erence 
al\'eatoire de vitesse angulaire et la phase sur l'orbite. Pour le moment on 
ne prend pas en compte les conditions d'\'eclairage. 

La d\'ependance de nombre des satellites rep\'er\'es de temps d'observation
est donn\'ee sur la \autoref{vladimir:stat}. 
 \begin{figure}[htp] \centering
      \includegraphics*[height=3in]{stat.png}
      \caption{
            \label{vladimir:stat}
  Proportion des satellites rep\'er\'es pour  $N = 100$ en vert, $N = 1000$ en bleu, $N = 5000$ en rouge, courbe th\'eorique
  en noir}
 \end{figure}
Qualitativement elle correspond bien \`a notre estimation th\'eorique m\^eme avec la loi
binomiale.  

Ce r\'esultat montre entre autres que l'hypoth\`ese de l'ind\'ependance 
d'observer le satellite pendant deux jours cons\'ecutifs est 
bien valable au d\'ebut d'observation. 
% La diff\'erence quantitative entre la probabilit\'e estim\'ee est simul\'ee
% est de 0,XX\%.  %L\`a je ne souviens plus...

Pour des temps plus longs, le comportement est plus subtil, comme on
peut le voir en échelle logarithmique (\autoref{vladimir:stat_log})

\begin{figure}[ht] \centering
      \includegraphics*[height=3in]{stat_log.png}
      \caption{
            \label{vladimir:stat_log}
  Proportion des satellites non-rep\'er\'es \`a l'\'echelle logarithmique $N = 100$ en vert, $N = 1000$ en bleu, $N = 5000$ en rouge, estimation th\'eorique
  de la loi binomiale en noir}
 \end{figure}

Un modèle plus avancé, prenant en compte les défauts d'ergodicité,
doit donc être utilisé. Pour le vérifier, il faudrait des simulations
plus longues et précises, que nous n'avons pas eu le temps de faire.  
\subsection{Condition d'\'eclairage et l'ergodicit\'e}  
Pour la deuxi\`eme s\'erie des simulations on ajoute la condition d'\'eclairage, c'est-\`a-dire 
si le d\'ebris est dans le champ de vue de t\'elescope il n'est consid\'er\'e comme rep\'er\'e
si le d\'ebris est \'eclair\'e est le t\'elescope ne l'est pas. Cela est g\'er\'e par la prise 
en compte de position du Soleil (\autoref{vladimir:sun}) --- voir la section 
« Conditions d'\'eclairement » pour les d\'etails. 

Le but de cette s\'erie des tests num\'eriques est \`a la fois de v\'erifier 
l'ind\'ependance (moyenne) de condition de visibilit\'e g\'eom\'etrique (position
relative de satellite et de t\'elescope) et d'\'eclairage (position
du Soleil) ainsi que d'\'etudier l'effet d'augmentation de nombre des
t\'elescopes. 

% {\small Calcul: cluster de l'ICJ Univ. Claude Bernard Lyon 1}
Le tableau suivant montre le nombre des d\'ebris observ\'es
\`a la fin de la simulation pour les diff\'erents temps d'observation et nombres des
t\'elescopes. Ici on fait varier tous les param\`etres du syst\`eme.

\begin{center}
\begin{tabular}{|c|c|c|c|} \hline
 D\'ebris & T\'el\'escopes & Temps & \% vus \\ 
 \hline \hline
 100 & 10 & $\sim$ 2 mois &82 \\
 \hline
 100 & 2 & $\sim$ 10 mois &89 \\
 \hline
 \hline
 1000 & 10 & $\sim$ 2 mois &85.5 \\
 \hline
 1000 & 5 & $\sim$ 4 mois &86.7 \\
 \hline
 1000 & 2 & $\sim$ 10 mois &87.7 \\
 \hline  
 \hline
 10000 & 10 & $\sim$ 2 mois &84.0 \\
 \hline
 \hline
 $10^5$ & 14 & $\sim$ 8 mois &  94.7 \\
 \hline
\end{tabular} 
\end{center}
  % {\small Calcul: cluster de l'ICJ Univ. Claude Bernard Lyon 1}
Ceci est coh\'erent avec la probabilit\'e journali\`ere d'observation 
   $p \approx 1\%$, ce qui est bien la valeur estim\'ee suivant l'hypoth\`ese 
   d'ind\'ependance pour les simulations courtes. 
   
   De plus on voit que la probabilit\'e d'observation \`a la fin de p\'eriode
   est la m\^eme quand le produit de temps d'observation et le nombre des t\'elescopes
   coïncide. On observe donc « le premier pas vers l'ergodicit\'e », quand la moyenne 
   temporelle est \'egale \`a la moyenne de l'ensemble. 
   
   \subsection{Impl\'ementation}
On va conclure la section de simulation par quelques mots sur l'impl\'ementation.
Tout d'abord l'algorithme utilis\'e est assez direct comme on \'etudie les solutions 
connues des \'equations de mouvement (\ref{vladimir:sat}, \ref{vladimir:tel}, \ref{vladimir:sun}). On a analys\'e la trajectoire enti\`ere de chaque d\'ebris m\^eme si cela n'est pas vraiment 
n\'ecessaire pour la statistique d'observation des d\'ebris.

On peut proposer plusieurs pistes de l'am\'elioration de la m\'ethode. 
%(non-impl\'ement\'ees, faute de temps).
La simulation utilis\'ee est assez simpliste au sens qu'on mod\'elise le ph\'enom\`ene m\^eme quand il
 ne nous int\'eresse pas, c'est-\`a-dire on \'etudie les parties longues des orbites pour lesquelles on est sûr 
qu'elle n'intersectent pas le champ de vue du t\'elescope. Pour \'eviter cela on peut estimer le temps de prochaine
intersection de l'orbite du d\'ebris avec la trajectoire du t\'elescope et \'etudier cette p\'eriode avec plus de pr\'ecision. 
C'est direct pour un seul t\'elescope et assez facile pour plusieurs. On remarque \'egalement que les simulations
pour les trajectoires des d\'ebris distincts sont ind\'ependantes,
c'est-\`a-dire que l'algorithme de calcul d'estimation de la
probabilit\'e d'observation est parall\'elisable avec un gain lin\'eaire. 
   
   


\section{Conclusion}
À partir d'une modélisation simple mais réaliste du système de trois
corps soleil-terre-débris, on aboutit à la conclusion que, si la
latitude du téléscope est compatible avec celle du débris, le
téléscope a deux fenêtres de 12s par jour de visibilité sur l'orbite
du débris. Pendant chaque fenêtre, il y a une probabilité d'environ
$1\%$ d'observer le débris. Si le téléscope est bien positionné sur la
terre, cette probabilité est, en moyenne, diminuée d'un facteur de 2
environ à cause des conditions d'éclairement.

Si les observations étaient indépendantes d'un jour sur l'autre, la
distribution du nombre d'observations d'un débris suivrait une loi
binomiale. L'étude d'un modèle jouet suggère que cette distribution se
rapproche plus d'une loi de Cauchy, dont les traînes sont très
épaisses.

Ces conclusions semblent être vérifiées par les simulations
numériques, qui montrent que l'observation de tous les débris est
problématique en temps long. Si cet effet se vérifie, il signifie
qu'il vaut mieux utiliser deux téléscopes pendant un an qu'un
téléscope pendant deux ans, une conclusion assez peu intuitive.

Notre étude, par sa courte durée, est limitée. En particulier, elle
souffre des limitations suivantes :
\begin{itemize}
\item on ne considère que des orbites circulaires. Néanmoins, notre
  modèle est adaptable;
\item on fait l'hypothèse que les débris tournent bien plus vite que
  la terre, c'est-à-dire que $T_{\text{débris}} \ll
  T_{\text{terre}}$. L'erreur est de $10\%$ pour les orbites basses
  (héliosynchrones), mais est très élevée pour les orbites hautes
  (géostationnaires) ;
\item on néglige totalement la dépendance de l'angle d'ouverture
  $\alpha$ par rapport à la position du téléscope, les conditions
  météo, etc. ;
\item nos simulations numériques sont naïves, ce qui conduit à un
  temps de calcul bien trop long par rapport au phénomène observé. Une
  méthode plus appropriée serait d'essayer de prédire les
  intersections de la zone d'observation du téléscope avec l'orbite du
  débris, et de n'intégrer les équations que dans cette zone.
\end{itemize}



\appendix

%\section{Code des simulations numériques}

% 
\section{Code des simulations numériques}

{\small
\noindent Fichier \underline{chaos.h}: 
Header
%\lstinputlisting{chaos.h}


\noindent Fichier \underline{engine.cpp}: 
déscription des procédures
%\lstinputlisting{engine.cpp}

\noindent Fichier \underline{traj.cpp}: 
IO et loop principal
%\lstinputlisting{traj.cpp}
}










\begin{thebibliography}{99}
\bibitem{kesten}{Kesten, H. (1960). Uniform distribution mod 1. The Annals of
    Mathematics, 71(3), 445-471.}
\end{thebibliography}
\end{document}
