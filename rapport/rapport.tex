\documentclass[11pt,a4paper,reqno]{amsart}
\setlength{\textwidth}{125mm}
\setlength{\textheight}{195mm}
\usepackage{fullpage}
\usepackage[utf8]{inputenc}
\usepackage[T1]{fontenc}
\usepackage{lmodern}
\usepackage[percent]{overpic}
\usepackage{pgf,tikz}
\usetikzlibrary{arrows}
\usepackage{enumerate}
\usepackage{amsmath, amssymb, amsthm}
\newcommand{\eps}{\varepsilon}
\author{Nicolas Bonnotte, Maxime Chupin, Tony Février,\\ Antoine Levitt,
  Benjamin Marteau, Vladimir Salnikov}

\begin{document}
\title{Observation de débris}

\maketitle
\begin{abstract}
  ...
\end{abstract}

% \tableofcontents
\section{Contexte}
Ce travail a été réalisé dans le cadre de la 4ème édition de la
Semaine d'Etude Mathématiques et Entreprises (SEME), qui s'est tenue à
dans les locaux de l'Institut Henri Poincaré (IHP) à Paris du Lundi 15
au Vendredi 19 octobre 2012. Le groupe souhaite remercier les
organisateurs de la SEME, ainsi que Astrium et plus précisément Max
Cerf pour avoir proposé ce sujet, que nous avons trouvé intéressant et
riche.

Les participants sont:
\begin{itemize}
\item Nicolas Bonnotte
\item Maxime Chupin
\item Tony Février
\item Antoine Levitt, doctorant à l'Université Paris Dauphine
\item Benjamin Marteau
\item Vladimir Salnikov
\end{itemize}
(rajoutez vos affiliations)
\section{Introduction}
\section{Observation du débris}
\section{Conditions d'éclairement}
\section{Ergodicité}
Considérons un téléscope et un débris. Si la latitude du téléscope le
permet, il croise l'orbite du débris deux fois par jour. Entre deux
observations séparées d'un temps $T$, le débris parcourt une
proportion $\theta$ de son orbite, où
\begin{align*}
  \theta(T) \approx \frac T {T_{\text{Débris}}}.
\end{align*}

Même si on prend en compte les corrections induites par la latitude,
l'inclinaison de l'orbite et la précession, l'important est que cet
angle $\theta(T)$ est toujours proportionel à $T$. Le fait qu'on fasse
deux observations par jour ne change pas fondamentalement le problème
qu'on étudie, et on est donc ramenés au modèle simplifié suivant:

\begin{center}
  \definecolor{qqwuqq}{rgb}{0,0.39,0}
\definecolor{ffqqqq}{rgb}{1,0,0}
\definecolor{qqqqff}{rgb}{0,0,1}
\begin{tikzpicture}[line cap=round,line join=round,>=triangle 45,x=1.0cm,y=1.0cm]
\clip(-2.46,-1.58) rectangle (5.82,5.82);
\draw [shift={(1,2)},color=qqwuqq,fill=qqwuqq,fill opacity=0.1] (0,0) -- (0:2) arc (0:74.05:2) -- cycle;
\draw [shift={(1,2)},color=qqwuqq,fill=qqwuqq,fill opacity=0.1] (0,0) -- (143.2:2) arc (143.2:161.57:2) -- cycle;
\draw [color=ffqqqq] (1,2) circle (3.16cm);
\draw (1,2)-- (-2,3);
\draw (1,2)-- (-1.53,3.89);
\draw (1,2)-- (4.16,2);
\draw (1,2)-- (1.87,5.04);
\begin{scriptsize}
\fill [color=qqqqff] (1,2) circle (1.5pt);
\fill [color=black] (4.16,2) circle (1.5pt);
\draw[color=black] (5.0,2.26) node {$\text{Débris à t}$};
\fill [color=black] (1.87,5.04) circle (1.5pt);
\draw[color=black] (3.54,5.3) node {$\text{Débris à }t+Tobs$};
\draw[color=qqwuqq] (2.22,2.6) node {$\theta$};
\draw[color=qqwuqq] (0.64,2.48) node {$\alpha$};
\end{scriptsize}
\end{tikzpicture}

  
  (refaire meilleure figure)
\end{center}

On pose
\begin{align}
  \label{modelejouet}
  \theta_{n} = (\theta_{0} + n \theta) \mod 1,
\end{align}
où $\theta_{0}$ est une phase initiale aléatoirement distribuée, et
$\theta$ est un paramètre. On considère qu'on a observation du débris
au jour $n$ si $\theta_{n} \in [0,\alpha]$, où $\alpha$ est
l'ouverture du téléscope. Les valeurs numériques donnent $\theta
\approx 14.1, \alpha \approx 0.1/360$, ce qui justifie l'approximation
$\alpha \ll \theta \mod 1$. On appelle $N(n)$ le nombre cumulé
d'observations du débris au jour $n$. On sait que si $\theta$ est
irrationnel, alors
\begin{align}
  \label{largenumbers}
  \lim_{n\to\infty} \frac{N(n)} n = \alpha.
\end{align}
(théorème de Birkhoff)

Ce théorème est un analogue de la loi des grand nombres : en temps
grand, la fréquence d'observation du débris est égale à la proportion
du cercle observée par le téléscope. Le problème est que ce théorème
ne nous apporte pas d'informations concrètes sur la vitesse de
convergence de cette limite, qui nous permettrait d'obtenir des
conclusions pratiques. 

Si au lieu du modèle déterministe \eqref{modelejouet} on utilise un
modèle probabiliste où $\theta_{n}$ est une variable aléatoire
distribuée uniformément entre 0 et 1, alors $N(n)$ suit la loi
binomiale $B(n,\alpha)$. Comme cette loi est connue, on peut extraire
les informations qui nous intéressent. Par exemple,
\begin{align}
  P(N(n) \geq 1) = 1 - (1-\alpha)^{n}
\end{align}

Le passage du modèle déterministe au modèle probabiliste pose
néanmoins problème. Même si le théorème \eqref{largenumbers} peut
laisser penser que, pour presque tout $\theta$, (et donc pour toutes
les valeurs physiques) on va observer un débris en un temps de l'ordre
de $1/\alpha$, ce n'est pas toujours le cas. Pour par exemple $\theta
= 1/2 + \eps$, avec $\eps$ petit, un débris placé avec un angle
initial $\theta_{0}$ proche mais plus petit que $1/2$ va être observé
au bout d'un temps de l'ordre de $1/\eps$.

Les simulations numériques que nous avons effectuées montrent que
l'approximation aléatoire est une bonne approximation, mais ne permet
pas de rendre compte de certains phénomènes (certains débris mettent
plus de temps à se laisser observer que dans le modèle aléatoire). La
question de la compréhension fine de ce phénomène est une question
ouverte intéressante du point de vue à la fois théorique (théorie
ergodique) et pratique (pour se prémunir de la possibilité de ne pas
observer certains débris aussi rapidement que ce que prédit l'approche
probabiliste).

\section{Résultats numériques}

\section{Conclusion}
Limitations, perspectives :
\begin{itemize}
\item Orbites circulaires (modèle adaptable)
\item $T_{\text{débris}} \ll T_{\text{terre}}$ : erreur de $10\%$ pour
  orbites basses (héliosynchrones), pas valable pour orbites hautes
  (géostationnaires)
\item Limites du modèle probabiliste, qui semble pourtant assez
  pertinent. Modèle effectif qui rendrait compte du synchronisme
  partiel des phénomènes ?
\item Pas de prise en compte fine du téléscope (quelle position
  choisir pour un bon $\alpha$ ?)
\item Simulations numériques naïves (timestep trop grand quand orbite
  en vue donc sous-estimation de la visibilité, et timestep trop petit
  quand orbite pas en vue donc temps de calcul beaucoup trop
  important). Essayer de prédire les intersections, et
  ``fast-forward'' de la simulation.
\end{itemize}

\end{document}
