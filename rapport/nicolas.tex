\renewcommand{\phi}{\varphi}

\newcommand{\TD}{T_\text{d}}
\newcommand{\omegaD}{\omega_\text{d}}
\newcommand{\iD}{i_\text{d}}
\newcommand{\altitudeD}{r}
\newcommand{\rayonD}{\rayonT+\altitudeD}

\newcommand{\TT}{T_\text{t}}
\newcommand{\omegaT}{\omega_\text{t}}
\newcommand{\iT}{i_\text{t}}
\newcommand{\rayonT}{R}

\newcommand{\referentiel}{\mathcal{R}}

\section{Observabilité d'une orbite}

\subsection{Référentiel et notations}

Supposons qu'au temps $t=0$, on puisse voir au centre du télescope une certaine orbite, et donc potentiellement les débris qui s'y trouveraient. Pendant combien de temps pourra-t-on encore la voir ? Et, quand elle aura disparue, pendant combien de temps l'aura-t-on vue au total ?

On se place dans un référentiel $\referentiel = (O,\vec{e}_x,\vec{e}_y,\vec{e}_z)$ dont le centre est le centre de la Terre, tel que $(O,\vec{e}_x,\vec{e}_y)$ soit le plan de l'équateur. Notons $\iT \in [-\frac{\pi}{2}, \frac{\pi}{2}]$ l'inclinaison du télescope vis-à-vis de l'équateur, et $\phi \in [0,\pi]$ sa latitude : $\iT = \frac{\pi}{2} - \phi$, et $\phi=0$ correspond au pôle Nord, $\phi = \pi$ au pôle Sud. En notant  $\rayonT$ le rayon de la Terre et $\omegaT = \frac{2\pi}{\TT}$ la pulsation du télescope, les coordonnées de ce dernier dans notre référentiel $\referentiel$  sont :
\[ 
\overrightarrow{OT} = \rayonT \vec{u} \qquad \text{avec} \qquad \vec{u}(t) = \left(\begin{array}{c}
\cos(\omegaT t) \cos(\iT) \\
\sin(\omegaT t) \cos(\iT) \\
\sin(\iT)
\end{array}
\right).
\]
Supposons que l'orbite soit un cercle de centre $O$, de rayon $\rayonD$, dans un plan incliné d'un angle $\iD \in [0, \frac{\pi}{2}]$ vis-à-vis de l'équateur et de direction normale
\[ 
\vec{n} =  \left(\begin{array}{c}
 \sin(\iD)\cos(\theta) \\
\sin(\iD)\sin(\theta) \\
 \cos(\iD) 
\end{array}
\right).
\]
%L'orbite se trouve donc dans le plan $(O, \vec{u}, \vec{v})$, avec
%\[ 
%\vec{u} =  \left(\begin{array}{c}
% -\sin(\theta) \\
%\cos(\theta) \\
%0 
%\end{array}
%\right)
%\qquad \text{et} \qquad
%\vec{v} = \vec{n} \wedge \vec{u} =  \left(\begin{array}{c}
% -\cos(\iD) \cos(\theta) \\
% -\cos(\iD)\sin(\theta) \\
%  \sin(\iD)
%\end{array}
%\right)
%\]
À $t=0$, le télescope coupe le plan de l'orbite, et cela entraîne $\vec{n}\cdot \vec{u} = 0$. L'angle $\theta$ doit donc vérifier la relation
\[ \cos(\theta) = - \frac{\sin(\iT)\cos(\iD)}{\cos(\iT)\sin(\iD)}.\]

On note $\alpha$ le demi-angle de d'ouverture du télescope, ou plutôt le demi-angle qui serait celui sous lequel un observateur placé au centre de la Terre verrait la même chose à l'altitude $\altitudeD$ que le télescope. Si $\alpha_0$ est l'ouverture réelle du télescope, on a (voir \autoref{nicolas:alpha}) :
\[ \alpha = \frac{\rayonT}{\rayonD} \alpha_0 \approx 0,1\text{\degres}.\]

\begin{figure}
\begin{center}
\scriptsize
\def\figurewidth{0.6\linewidth}
\newlength{\unit}
\newlength{\width}
\setlength{\width}{\figurewidth}
\setlength{\unit}{0.0926\width}
%\setlength{\unit}{1cm}
%\setlength{\unit}{\texwidth}

\definecolor{xdxdff}{rgb}{0.49,0.49,1}
\definecolor{qqwuqq}{rgb}{0,0.39,0}
\definecolor{qqqqcc}{rgb}{0,0,0.8}
\definecolor{qqqqff}{rgb}{0,0,1}
\begin{tikzpicture}[line cap=round,line join=round,>=triangle 45,x=\unit,y=\unit]
\clip(4.11,-0.63) rectangle (10.8,5.62);
\draw [shift={(5,3)},color=qqwuqq,fill=qqwuqq,fill opacity=0.1] (0,0) -- (49.79:0.75) arc (49.79:90:0.75) -- cycle;
\draw [shift={(5,0)},color=qqwuqq,fill=qqwuqq,fill opacity=0.1] (0,0) -- (72.58:0.75) arc (72.58:90:0.75) -- cycle;
\draw [color=red, densely dotted] (5,0) circle (5\unit);
\draw [color=red] (5,5) arc (90:72:5\unit);
\draw [color=qqqqcc] (5,0) circle (3\unit);
\draw[dashed] (5,3)-- (5,0);
\draw (5,5)-- (5,3);
\draw[dashed] (5,0)-- (6.5,4.77);
\draw (5,3)-- (6.5,4.77);
\draw (4.5,4.12) node[anchor=north west] {$r$};
\draw (4.45,1.84) node[anchor=north west] {$R$};
\begin{scriptsize}
\fill [color=black] (5,0) circle (1.5pt);
\fill [color=red] (5,5) circle (1pt);
\fill [color=black] (5,3) circle (1.5pt);
\draw[color=qqqqff] (4.75,3.25) node {$T$};
\fill [color=red] (6.5,4.77) circle (1pt);
\draw[color=qqwuqq] (5.625,3.25) node {$\alpha_0$};
\draw[color=qqwuqq] (5.375,0.25) node {$\alpha$};
\fill [color=black] (5.75,4.94) circle (1.5pt);
\draw[color=red] (6.07,5.15) node {$D$};
\end{scriptsize}
\end{tikzpicture}

\caption{Rapport entre les angles $\alpha$ et $\alpha_0$} \label{nicolas:alpha}
\end{center}
\end{figure}


\subsection{Calcul approché de la durée journalière d'observation}

Supposons que l'orbite apparaisse dans la visée du télescope avec un angle $\psi$ vis-à-vis de l'horizontale. Elle balaye la fenêtre d'observation avec une vitesse $\omegaT \sin(\psi)$ (voir \autoref{nicolas:fenetre}). Il lui faut donc un temps $\tau_0$ pour traverser entièrement celle-ci, avec $\tau_0$ donné par la formule
\[ \tau_0 = \frac{2\alpha}{\omegaT \sin(\psi)}.\]
Cet angle $\psi$ est l'angle entre le vecteur vitesse du télescope $T$ et celui d'un débris $D$ situé sur l'orbite. 
%On connait déjà $\overrightarrow{OT}$ ; quant à $\overrightarrow{OD}$, on sait qu'il se déplace dans un plan $(O, \vec{f}_1, \vec{f}_2)$ perpendiculaire à $\vec{n}$ sur un cercle de rayon $\rayonD$ de centre $O$ et de vitesse angulaire $\omegaD = \frac{2\pi}{\TD}$, et que $O$, $T$ et $D$ sont alignés pour $t=0$. Ceci nous suffit pour obtenir :
%\[ \overrightarrow{OD} = (\rayonD)\vec{v} \qquad \text{avec} \qquad \vec{v} = \cos(\omegaD t) \vec{f}_1 + \sin(\omegaD t) \vec{f}_2\]
%si du moins
%\[ 
%\vec{f}_1 =  \left(
%	\begin{array}{c}
%		\cos(\iT) \\
%		0\\
%		\sin(\iT)
%	\end{array}
%\right) \quad \text{et} \quad \vec{f}_2 =  \frac{1}{\cos(\iT)} \left(
%	\begin{array}{c}
%		\sin(\iT) \sqrt{\sin(\iD)^2 - \sin(\iT)^2} \\
%		\cos(\iD) \\
%		-\cos(\iT) \sqrt{\sin(\iD)^2 - \sin(\iT)^2}
%	\end{array}
%\right).
%\]
%Par conséquent, l'angle $\psi$ est donné par la formule
On peut montrer que
\[ 
\cos(\psi) = \frac{\cos(\iD)}{\cos(\iT)}.
\]
Puisqu'il y a deux observations par jour, la durée journalière d'observation de l'orbite est
\[ \tau = 2\tau_0 = \frac{4\alpha\cos(\iT)}{\sqrt{\sin(\iD)^2 - \sin(\iT)^2}}.\]
Malheureusement, cette formule n'a probablement pas de sens si $\iD$ et $\iT$ sont trop rapprochés, ou si $\iT$ est trop proche de $\frac{\pi}{2}$. C'est pourquoi nous allons maintenant essayer une autre méthode, exacte celle-ci.

\begin{figure}
\begin{center}
\scriptsize
\def\figurewidth{0.6\linewidth}
%\newlength{\unit}
%\newlength{\width}
\setlength{\width}{\figurewidth}
\setlength{\unit}{0.3\width}



\definecolor{qqzzqq}{rgb}{0,0.6,0}
\definecolor{ffqqqq}{rgb}{1,0,0}
\definecolor{qqqqcc}{rgb}{0,0,0.8}
\definecolor{uququq}{rgb}{0.25,0.25,0.25}

\begin{tikzpicture}[line cap=round,line join=round,>=triangle 45,x=\unit,y=\unit]
\draw[color=black, dashed] (-1,0) -- (1,0);
%\foreach \x in {-1,-0.5,0.5,1}
%\draw[shift={(\x,0)},color=black] (0pt,2pt) -- (0pt,-2pt) node[below] {\footnotesize $\x$};
%\draw[color=black] (0,-1.36) -- (0,1.28);
%\foreach \y in {-1,-0.5,0.5,1}
%\draw[shift={(0,\y)},color=black] (2pt,0pt) -- (-2pt,0pt) node[left] {\footnotesize $\y$};
%\draw[color=black] (0,0.125) node[right] {\footnotesize $O$};
\clip(-1.44,-1.36) rectangle (1.31,1.28);
\draw [shift={(-1.25,-1)},color=qqqqcc,fill=qqqqcc,fill opacity=0.1] (0,0) -- (0:0.18) arc (0:75.07:0.18) -- cycle;
\draw [->] (-1.25,-1) -- (-0.85,-1);
\draw [->] (-1.25,-1) -- (-1.05,-0.25);
\draw [color=red,domain=-.486:0.0133] plot(\x,{(--0.19--0.75*\x)/0.2});
\draw [color=red,densely dotted,domain=-0.0368:-.5295] plot(\x,{(--0.19--0.75*(\x+0.05))/0.2});
\draw [color=red,dotted,domain=-0.0876:-.5720] plot(\x,{(--0.19--0.75*(\x+0.1))/0.2});
\draw [color=red,loosely dotted,domain=-.13926:-.6138] plot(\x,{(--0.19--0.75*(\x+0.15))/0.2});
%\draw [color=qqzzqq,domain=-1.44:1.31] plot(\x,{(-0-0.2*\x)/0.75});
\draw [->, color=red] (-0.43,-0.67) -- (-0.06,-0.77);
%\draw [->] (-0.43,-0.67) -- (-0.2,0.18);
\begin{scriptsize}
\fill [color=black] (0,0) circle (1.5pt);
%\draw[color=uququq] (0.05,0.08) node {$O$};
\draw[color=black] (-0.87,-0.92) node {$\omegaT$};
\draw[color=black] (-1,-0.61) node {$\omegaD$};
\draw[color=qqqqcc] (-1.15,-0.92) node {$\psi$};
\draw[color=black] (0.13,-0.66) node {$\omegaT \sin(\psi)$};
%\draw[color=black] (0.15,-0.3) node {$\omegaD + \omegaT \cos(\psi)$};
\draw [line width=1pt] (0,0) circle (1);
\end{scriptsize}
\end{tikzpicture}
\caption{Fenêtre d'observation} \label{nicolas:fenetre}
\end{center}
\end{figure}



\subsection{Calcul exact de la durée journalière d'observation}

On rappelle que $\alpha$ désigne le demi-angle d'observation en équivalent pour un observateur situé au centre de la Terre. L'orbite est donc visible si et seulement si
\[ -\sin(\alpha) \leq \vec{n} \cdot \vec{u} \leq \sin(\alpha).\]
Notons $t_\pm$ les deux instants les plus proches de $t=0$ tels que
\[  \vec{n} \cdot \vec{u}(t_+) = \sin(\alpha) \qquad \text{et} \qquad  \vec{n} \cdot \vec{u}(t_-) = -\sin(\alpha).\] 
Puisque $\alpha$ est petit, on a donc :
\begin{multline*}
\cos(\omegaT t_\pm) \cos(\iT) \sin(\iD) \cos(\theta) + \sin(\omegaT t_\pm) \cos(\iT)\sin(\iD)\sin(\theta)
 = \pm\alpha -  \sin(\iT)\cos(\iD).
\end{multline*}
Ceci entraîne
\[ \cos(\theta - \omegaT t_\pm) = \cos(\theta) \pm  \frac{\alpha}{\cos(\iT)\sin(\iD)}.\]
Puisque $t_\pm$ est petit, 
\[ \sin(\theta) \omegaT t_\pm - \frac{1}{2} \cos(\theta) \left(\omegaT t_\pm\right)^2 = \pm  \frac{\alpha}{\cos(\iT)\sin(\iD)},\]
et on obtient donc
\[ \omegaT t_\pm = A - \sqrt{A^2 \mp \frac{\alpha\cos(\iT)\sin(\iD)}{\sin^2(\iT)\cos^2(\iD)}} \qquad \text{avec} \qquad A=\frac{\cos(\iT)\sin(\iD)\sqrt{\sin^2(\iD)-\sin^2(\iT)}}{\sin^2(\iT)\cos^2(\iD)}\]
Le taux d'observation sur une journée est  $ \tau = 2(t_+ - t_-)$, en convenant que $t_\pm = 0$ lorsque la formule donnant $t_\pm$ n'a pas de sens. On a :
\[ \tau = \frac{2}{\omegaT} \left(  \sqrt{A^2 + \frac{\alpha\cos(\iT)\sin(\iD)}{\sin^2(\iT)\cos^2(\iD)}} - \sqrt{A^2 - \frac{\alpha\cos(\iT)\sin(\iD)}{\sin^2(\iT)\cos^2(\iD)}} \right).\]
En particulier,
\begin{align*}
  \tau =  &\frac{2}{\omegaT}\sqrt{\frac{\alpha}{\cos(\iD)\sin(\iD)}} & \text{si $\iD = \iT$}, \\
  \tau =  &\frac{2}{\omegaT}\frac{\alpha}{\sqrt{\sin(\iD)^2 - \sin(\iT)^2}} & \text{si $\iD > \iT \gg \alpha$}.
\end{align*}%

\begin{figure}
\scriptsize
\begin{center}
\def\svgwidth{0.5\linewidth}
 \input{figures/observabilite3.pdf_tex}
 \vspace*{-2em}
\caption{Durée journalière d'observation (en minutes)} \label{nicolas:observabilite}
\end{center}
\end{figure}%

La \autoref{nicolas:observabilite} montre le nombre de minutes par jour où l'orbite est visible. Cette durée ne dépend que de $\iT$ et de $\iD$. Lorsque $\iD=\iT=0$ ou $\iD=\iT=\frac{\pi}{2}$, l'orbite est en permanence visible : le premier cas correspond au télescope situé sur l'équateur avec une orbite équatoriale, le second au télescope au pôle Nord avec une orbite polaire. Lorsque $\iD < \iT$, l'orbite n'est pas assez inclinée et n'est donc jamais visible par le télescope, situé lui à une latitude trop élevée. Lorsque $\iD = \frac{\pi}{2}$ et $\iT = \frac{\pi}{4}$, c'est-à-dire pour une orbite polaire et un télescope à la latitude de Bordeaux, l'orbite est visible 96 secondes par jour. Lorsque $\iD = \frac{\pi}{2}$ et $\iT = 0$, c'est-à-dire pour une orbite polaire et un télescope à l'équateur, l'orbite n'est plus visible que 48 secondes par jour. 


\subsection{Partie de l'orbite observée chaque jour}

Lorsque l'on voit l'orbite durant un temps $\mathrm{d}t$, celle-ci tournant avec une vitesse $\omegaD$, on observe en fait une proportion $\mathrm{d}p = \frac{\omegaD}{2\pi} \mathrm{d} t = \frac{\mathrm{d}t}{\TD}$. La proportion de l'orbite observée par jour est donc
\[ p = \frac{\omegaD \tau}{2\pi} = \frac{\tau}{\TD}.\]
Si $\TD = 2 \text{h}$, pour une orbite polaire et un télescope à l'équateur on observe $p \approx 1 \%$ de l'orbite par jour. En supposant donc que l'on n'observe jamais deux fois les mêmes débris, et que ceux-ci sont répartis uniformément  sur l'orbite, on peut donc espérer observer $1 \%$ des débris par jours.


