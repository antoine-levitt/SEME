\section{Observation de l'orbite}

\frame{
  \frametitle{Ouverture angulaire}

On ignore l'éclairage pour l'instant


  \begin{center}
  \resizebox{0.6\textwidth}{!}{ \newlength{\unit}
\newlength{\width}
\setlength{\width}{\figurewidth}
\setlength{\unit}{0.0926\width}
%\setlength{\unit}{1cm}
%\setlength{\unit}{\texwidth}

\definecolor{xdxdff}{rgb}{0.49,0.49,1}
\definecolor{qqwuqq}{rgb}{0,0.39,0}
\definecolor{qqqqcc}{rgb}{0,0,0.8}
\definecolor{qqqqff}{rgb}{0,0,1}
\begin{tikzpicture}[line cap=round,line join=round,>=triangle 45,x=\unit,y=\unit]
\clip(4.11,-0.63) rectangle (10.8,5.62);
\draw [shift={(5,3)},color=qqwuqq,fill=qqwuqq,fill opacity=0.1] (0,0) -- (49.79:0.75) arc (49.79:90:0.75) -- cycle;
\draw [shift={(5,0)},color=qqwuqq,fill=qqwuqq,fill opacity=0.1] (0,0) -- (72.58:0.75) arc (72.58:90:0.75) -- cycle;
\draw [color=red, densely dotted] (5,0) circle (5\unit);
\draw [color=red] (5,5) arc (90:72:5\unit);
\draw [color=qqqqcc] (5,0) circle (3\unit);
\draw[dashed] (5,3)-- (5,0);
\draw (5,5)-- (5,3);
\draw[dashed] (5,0)-- (6.5,4.77);
\draw (5,3)-- (6.5,4.77);
\draw (4.5,4.12) node[anchor=north west] {$r$};
\draw (4.45,1.84) node[anchor=north west] {$R$};
\begin{scriptsize}
\fill [color=black] (5,0) circle (1.5pt);
\fill [color=red] (5,5) circle (1pt);
\fill [color=black] (5,3) circle (1.5pt);
\draw[color=qqqqff] (4.75,3.25) node {$T$};
\fill [color=red] (6.5,4.77) circle (1pt);
\draw[color=qqwuqq] (5.625,3.25) node {$\alpha_0$};
\draw[color=qqwuqq] (5.375,0.25) node {$\alpha$};
\fill [color=black] (5.75,4.94) circle (1.5pt);
\draw[color=red] (6.07,5.15) node {$D$};
\end{scriptsize}
\end{tikzpicture}
}
  \end{center}
  
 Le téléscope observe un cône d'angle $\tilde \alpha$ autour du zénith

$\rightarrow$ équivalent à un observateur au centre de la terre avec un
    angle $\alpha \approx \frac{r}{R+r}\tilde \alpha\approx 0.1^{\circ}$

}


\begin{frame}{Quelle proportion des débris voit-on ?}

On suppose l'orbite uniformément remplie. Quelle proportion des débris voit-on à chaque observation ?

\bigskip

On note :
\begin{itemize}
\item $\omega_D$ la vitesse (angulaire) de déplacement des débris
\item $\omega_T$ celle du télescope $T$
\item $\theta$ l'angle entre la trajectoire du télescope et celle des débris:
\[ \theta = \arccos\left(\frac{\cos{i_D}}{\sin(\varphi)}\right) \]
\item $\varphi$ latitude de $T$ ($\varphi = 0$ au pôle Nord, $\frac{\pi}{2}$ à l'équateur)
\item $i_D$ inclinaison de l'orbite ($i_D = 0$ sur le plan de l'équateur, $\frac{\pi}{2}$ pour une orbite polaire)
\end{itemize}

\bigskip

\end{frame}

\definecolor{qqzzqq}{rgb}{0,0.6,0}
\definecolor{ffqqqq}{rgb}{1,0,0}
\definecolor{qqqqcc}{rgb}{0,0,0.8}
\definecolor{uququq}{rgb}{0.25,0.25,0.25}
\begin{frame}{Quelle proportion des débris voit-on ?}
\begin{center}
\begin{tikzpicture}[line cap=round,line join=round,>=triangle 45,x=3cm,y=3cm]
\draw[color=black] (-1.44,0) -- (1.31,0);
%\foreach \x in {-1,-0.5,0.5,1}
%\draw[shift={(\x,0)},color=black] (0pt,2pt) -- (0pt,-2pt) node[below] {\footnotesize $\x$};
\draw[color=black] (0,-1.36) -- (0,1.28);
%\foreach \y in {-1,-0.5,0.5,1}
%\draw[shift={(0,\y)},color=black] (2pt,0pt) -- (-2pt,0pt) node[left] {\footnotesize $\y$};
\draw[color=black] (0pt,-10pt) node[right] {\footnotesize $0$};
\clip(-1.44,-1.36) rectangle (1.31,1.28);
\draw [shift={(-1.25,-1)},color=qqqqcc,fill=qqqqcc,fill opacity=0.1] (0,0) -- (0:0.18) arc (0:75.07:0.18) -- cycle;
\draw(0,0) circle (1);
\draw [->] (-1.25,-1) -- (-0.85,-1);
\draw [->] (-1.25,-1) -- (-1.05,-0.25);
\draw [color=ffqqqq,domain=-1.44:1.31] plot(\x,{(--0.19--0.75*\x)/0.2});
\draw [color=qqzzqq,domain=-1.44:1.31] plot(\x,{(-0-0.2*\x)/0.75});
\draw [->] (-0.43,-0.67) -- (-0.06,-0.77);
\draw [->] (-0.43,-0.67) -- (-0.2,0.18);
\begin{scriptsize}
\fill [color=uququq] (0,0) circle (1.5pt);
%\draw[color=uququq] (0.05,0.08) node {$O$};
\draw[color=black] (-0.87,-0.92) node {$\omega_T$};
\draw[color=black] (-1,-0.61) node {$\omega_D$};
\draw[color=qqqqcc] (-1.15,-0.92) node {$\theta$};
\draw[color=black] (0.13,-0.66) node {$\omega_T \cos(\theta)$};
\draw[color=black] (0.1,-0.2) node {$\omega_D + \omega_T sin(\theta)$};
\end{scriptsize}
\end{tikzpicture}

{\scriptsize \textcolor{red}{Rouge} : orbite, \textcolor{qqzzqq}{Vert} : direction apparente de déplacement de l'orbite }
\end{center}
\end{frame}

\begin{frame}{Quelle proportion des débris voit-on ?}
L'orbite semble se déplacer à une vitesse $\omega_T \sin(\theta)$, la fenêtre d'observation a un diamètre angulaire $2\alpha$ : on observe donc l'orbite durant
\[ T_{\text{obs}} = \frac{2\alpha}{\omega_T \sin(\theta)} \]

\bigskip

Puisque $\omega_D \gg \omega_T$, par chaque point de l'orbite visible on voit passer durant $\mathrm{d}t$ un nombre $\omega_D \mathrm{d}t$ de débris. La proportion totale observée est donc
\[ \color{red} T_\text{obs} \omega_D =  \frac{2\alpha\omega_D}{\sin(\theta) \omega_T }  \]
\end{frame}

